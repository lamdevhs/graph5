\input{/mnt/lab/cs/latex/format/article-spread.tex}

\input{/mnt/lab/cs/latex/base.tex}
\input{/mnt/lab/cs/latex/macros.tex}

\input{/mnt/lab/cs/latex/math/base.tex}
\input{/mnt/lab/cs/latex/math/macros.tex}
\input{/mnt/lab/cs/latex/math/better-sections.tex}

\author{Nathanael Bayard --- L3 DLMI --- Algorithmique des Graphes}
\title{Mini-Projet \\ Problème de Couplage (Matching)}
\date{\today}
    
\begin{document} \maketitle
\section{Présentation du Problème et Théorème de Hall}
On peut définir le problème de façon abstraite, de la manière suivante :

\SEP\jdefi On appelle \emph{problème de couplage} la donnée d'un triplet $P = (X,Y,\phi)$ tel que :
\mathL
\item $X$ et $Y$ soient des ensembles finis quelconques non vides.
\item $\phi \subseteq X \times Y$
\endL
Un élément $y \in Y$ et dit compatible avec un élément $x \in X$ lorsque $(x,y) \in \phi$. On introduit la notation suivante :
\[ \forall x \in X, \quad \phi x := \set { y \in Y \telq (x,y) \in \phi} \]
que l'on étend aux sous-parties de $X$ :
\[ \forall B \subseteq X, \quad \phi B := \bigcup_{x \in B} \phi x = \set{y \in Y \telq \exists x \in B,  (x,y) \in \phi}\]
$\phi x$ est donc l'ensemble des éléments de $Y$ compatibles avec $x$, $\phi B$ est l'ensemble des éléments de $Y$ compatibles avec au moins un élément de $B$.
\SEP

On définit alors une de solution au problème de la manière suivante :
\SEP\jdefi On appelle \emph{couplage} pour le problème de couplage $P = (X,Y,\phi)$ toute application bijective $f : A \subseteq X \to B \subseteq Y$, qui ``respecte'' $\phi$ dans le sens que :
\[ \forall x \in A, \quad f(x) \in \phi x \]
De façon équivalente, cela signifie que le graphe $\Gam_f = \set{(x,f(x)) \pour x \in A}$ de $f$ (au sens d'un graphe d'application) doit être inclus dans $\phi$. On appellera \emph{étendue} du couplage $f$, le cardinal de son graphe d'application, qui vaut aussi en particulier $\abs{\Img f}$.

On appelle \emph{couplage parfait} pour $P$ un couplage $f$ dont l'ensemble de définition soit $X$ tout entier, et l'ensemble d'arrivée soit $Y$. L'existence d'un couplage parfait entraîne évidemment que nécessairement $\abs X = \abs Y$.

On appelle \emph{couplage maximum} pour le problème $P$, tout couplage $f$ de $P$ tel que pour tout autre couplage $g$, $\abs{\Img g} \leq \abs{\Img f}$. On peut aussi l'interpréter en disant que $f$ est un maximum au sens de son étendue parmi tous les couplages de $P$ qui existent.
\SEP

Tout couplage parfait est bien sûr maximum, mais la réciproque n'est à priori pas vraie. Le Théorème de Hall donne une condition nécessaire et suffisante sur $P$ pour qu'il admette un couplage parfait.

\SEP\jtheoz{Hall} Soient $X$,$Y$ deux ensembles finis non vides, $\phi \subseteq X \times Y$, et $P = (X,Y,\phi)$ le problème de couplage associé. Alors $P$ admet un couplage parfait si et seulement si 
\mathL
\item $\phi$ respecte la condition suivante, dite \emph{condition de couplage} :
\[ \forall B \subseteq X, \quad \abs B \leq \abs{ \phi B } \tag{1} \]
\item $\abs X = \abs Y$
\endL
\SEP\jpreuve

    Prouvons la nécessité de ces deux conditions. Remarquons en premier lieu que si $f : A \subseteq X \to \Img f $ est un couplage pour $P$, alors comme $f(x) \in \phi x$ pour tout $x \in A$, on a l'inclusion :
    \[ \forall B \subseteq A, \quad f(B) \subseteq  \bigcup_{x \in B} \phi x = \phi B \]
    En particulier si $P$ admet un couplage parfait $f : X \to Y$, l'inclusion ci-dessus entraîne bien la condition énoncée par le Théorème de Hall :
    \[ \forall B \subseteq X, \quad \abs B = \abs{f(B)} \leq \abs{ \phi B } \]
    La première égalité venant de l'injectivité de $f$. Sa bijectivité entraîne aussi évidemment $\abs X = \abs Y$.
    \medskip
    
    Prouvons maintenant que ces conditions sont suffisantes pour obtenir l'existence d'un couplage parfait. Par hypothèse, $\phi$ respecte donc la condition (1). Soit $\gam \subseteq \phi$ respectant aussi la condition (1), et minimal au sens que pour tout $\xi \subset \gam$, $\xi$ ne respecte pas la condition (1).
    
    Soit $z \in X$. Puisque $\gam$ respecte la condition de couplage, on a l'inégalité $\abs {\gam z} = \abs {\gam \set z} \geq \abs{\set{z}} = 1$.   Montrons que $\abs{\gam z} = 1$ par l'absurde.
    
    Supposons que $\set{u,v} \subseteq \gam z$, avec $u \neq v$. On définit $\lam_u := \gam \minus \set{(z,u)}$. On définit symétriquement $\lam_v := \gam \minus \set{(z,v)}$. Remarquons que
    \[ \forall B \subseteq X \minus \set z, \quad \lam_u B = \lam_v B = \gam B \tag{2} \]
    
    Puisque $\lam_u \subset \gam$ et idem pour $\lam_v$, alors, par construction de $\gam$, ni $\lam_u$ ni $\lam_v$ ne peuvent vérifier la condition (1), et alors :
    
    \[ \exists B_u \subseteq X, \quad \abs{B_u} > \abs{W_u} \avec W_u := \lam_u B_u \]
    \[ \exists B_v \subseteq X, \quad \abs{B_v} > \abs{W_v} \avec W_v := \lam_v B_v \]
    
    Nécessairement, au vu de (2), $z$ appartient à $B_u$ et aussi à $B_v$. Définissons $B := (B_u \cap B_v) \minus \set{z}$, $N := \gam B$, et $W := W_u \cap W_v$.
    
    Soit $a \in B$. Alors $a \in B_u$ et comme $\gam a = \lam_u a$ (car $a \neq z$),
    \[ \gam a \subseteq  \bigcup_{x \in B_u} \lam_u x = W_u \]
    D'où $N \subseteq W_u$, et de même $N \subseteq W_v$ pour les mêmes raisons : finalement, $N \subseteq W$.
    
    Nous aurons aussi besoin du résultat suivant :
    \[ W_u \cup W_v = \gam (B_u \cup B_v) \tag{3} \]
    L'inclusion $\subseteq$ est évidente du fait que $\gam$ contient $\lam_u$ et $\lam_v$. Pour l'inclusion réciproque, la seule chose à vérifier est que $u$ et $v$ soient bien dans le membre de gauche, puisqu'ils sont dans le membre de droite. Or, $u \in \lam_v z \subseteq \lam_v B_v = W_v$ puisque $z \in B_v$, et idem $v \in W_u$. D'où l'égalité (3).
    
    Comme $\gam$ respecte la condition de couplage, on en déduit que
    \[ \abs{W_u \cup W_v} = \abs{\gam (B_u \cup B_v) } \geq \abs{B_u \cup B_v} \tag{4} \]
    
    Finalement :
    \[ \begin{aligned}
        \abs{N} = \abs{\gam B} &\leq \abs{W}  \car N \subseteq W \\
        & = \abs{ W_u \cap W_v } = \abs {W_u} + \abs{W_v} - \abs{W_u \cup W_v}\\
        & \leq (\abs {B_u} - 1)  + (\abs{B_v} - 1) - \abs{B_u \cup B_v} \\
        &    \qquad  \puisque \abs{W_u} < \abs{B_u} \et \abs{W_v} < \abs{B_v} \et \text{(4)} \\
        & =    \abs {B_u \cap B_v} - 2 \\
        & =    \abs {(B_u \cap B_v) \minus \set{z}} - 1 \\
        & =    \abs{B} - 1
    \end{aligned} \]
    D'où finalement $\abs{\gam B} < \abs{B}$ : absurde puisque $\gam$ respecte la condition de couplage (1).
    
    On vient donc de prouver que $\forall z \in X, \abs{\gam z} = 1$.
    
    On peut définir une application $f : X \to Y$ qui à $x \in X$ lui associe l'unique $y \in Y$ tel que $\gam x = \set{y}$. Reste à démontrer la bijectivité de $f$ : si $f(x) = f(x') = y$, c'est que $\gam x = \gam x' = \set{y}$, et alors comme $\gam$ respecte la condition de couplage :
    \[ \abs{\set{x,x'}} \leq \abs{\gam\set{x, x'}} = \abs{\set{y}} = 1 \]
    D'où $x = x'$, et $f$ est bien injective, donc bijective puisque d'après (ii), $\abs X = \abs Y$. $f$ est donc bien un couplage parfait pour $P$, ce qui achève cette démonstration.
 \qed\SEP
 
 De nombreux problèmes concrets peuvent être modélisés par un problème de couplage. L'objectif est souvent de construire une bijection $f : X \to Y$ avec des contraintes sur les valeurs que peuvent prendre $f(x)$ ou $f\-(y)$, contraintes dépendant de $x$ ou de $y$.
 
 Il est à remarquer que la notion de problème de couplage symétrique par rapport à $X$ et $Y$.
 \SEP\jdefi
    Soit $P = (X,Y,\phi)$ un problème de couplage. Alors on définit son \emph{symétrique} $P\-$ comme le problème de couplage $P\- := (Y,X,\phi\-)$, avec :
    \[\phi\- := \set{ (y,x) \in Y \times X \telq (x,y) \in \phi } \]
 \SEP
 
 Si $f : A \subseteq X \to \Img f$ est un couplage parfait (respectivement maximum) pour $P = (X,Y,\phi)$ alors $f\-$ est un couplage parfait (respectivement maximum) pour $P\-$. En effet, $f\-$ sera bien une bijection de $\Img f \subseteq Y$ dans $A \subseteq X$, et il est clair que si $\Gam_f \subseteq \phi$, alors $\Gam_{f\-} \subseteq \phi\-$. L'\emph{étendue} de $f$ est bien sûr la même que celle de $f\-$.
 
 \section{Modélisation par un Graphe}
 
 Ce type de problème se prête naturellement bien à la modélisation par un graphe (au sens de la théorie des graphes). En fait cela nous permet d'éviter de distinguer les problèmes symétriques $P$ et $P\-$.
 
 \SEP\jdefi On appelle \emph{graphe du problème de couplage} $P = (X,Y,\phi)$, le graphe non orienté, simple, biparti $G := (V, E)$, avec $V := X \cup Y$ (on supposera ici $X$ et $Y$ disjoints) et $E := \phi$ (en identifiant 2-uplets de sommets, et arêtes).
 
 Ici comme $G$ est non orienté, la notation $(x,y)$ désigne donc une arête reliant $x$ à $y$, et correspond plus rigoureusement au doubleton $\set{x,y}$.
 
 Un \emph{couplage} du graphe $G$ correspondra à un ensemble d'arêtes $M \subseteq E$ tel qu'il existe un couplage $f : A \subseteq X \to \Img f$ de $P$, tel que le graphe d'application $\Gam_f$ de $f$ soit $M$, c'est-à-dire :
 \[ M = \Gam_f = \set{ (x,f(x)) \pour x \in A } \]
 
 En termes plus graphiques, du fait de l'injection de $f$, $M$ correspond à un ensemble d'arêtes de $G$ deux à deux disjoints par sommets. En particulier, chaque sommet de $G$ admet au plus un voisin dans le graphe partiel $(V, M)$.
 
 Un \emph{couplage parfait} de $G$ correspondra similairement à un ensemble $M = \Gam_f \subseteq E$ avec $f : X \to Y$ un couplage parfait pour $P$. L'existence d'un couplage parfait pour $G$ implique bien sûr $\abs X = \abs Y$.
 
 Un \emph{couplage maximum} pour $G$ correspondra à un ensemble $M = \Gam_f \subseteq E$ avec $f : A \subseteq X \to \Img f$ un couplage maximum pour $P$.
 
 Cela correspond aussi exactement à un couplage $M$ pour $G$ de cardinal maximal, puisque le nombre d'arêtes dans $M$ est exactement égal à l'étendue du couplage de $P$ qui lui correspond. Un couplage maximum $M$ est donc un ensemble d'arêtes de $G$ deux à deux sommet-disjointes et qui soit maximal au sens du cardinal.
 \SEP
 
 Les graphes des problèmes $P$ et $P\-$ sont exactement égaux, et tout couplage (resp. maximum, resp. parfait) de l'un est un couplage (resp. maximum, resp. parfait) pour l'autre.
 
 \section{Algorithme de Hopcroft-Karp}
 
 Le problème qui consiste à trouver un couplage maximum pour un graphe biparti simple $G = (V = X \cup Y,E)$ peut être résolu en temps polynômial grâce à l'algorithme de Hopcroft–Karp. Plus exactement, la complexité dans le pire cas est en $O(m \sqrt n)$ avec $n := \abs V$ et $m := \abs E$.
 
 \subsection{L'idée}
 
 \SEP\jdefi Soit $G = (V = X \cup Y, E)$ le graphe d'un problème de couplage, et $M \subseteq E$ un couplage de $G$. On dit que $v \in V$ est libre par rapport à $M$ s'il est isolé dans $(V,M)$, c'est-à-dire s'il n'est pas l'extrémité d'une arête dans $M$.
 
 On dira que $C$ est un \emph{chemin d'augmentation} de $M$ dans $G$, lorsque $C$ est un chemin de $G$, simple et élémentaire, de longueur non nulle, dont les deux sommets d'extrémité sont libres par rapport à $M$, et dont la suite des arêtes traversées est en alternance dans $M$ et dans $E \minus M$.
 \SEP
 
 Comme exemple trivial, un chemin d'augmentation ``minimaliste'' peut très bien être une unique arête dans $E\minus M$ reliant deux sommets libres.
 
 Tout sommet faisant partie d'un chemin d'augmentation de $M$ est, ou bien une des extrémités libres du chemin, ou bien n'est pas libre, puisqu'en effet il est dans ce cas l'extrémité d'exactement une arête dans $M$.
 
 Du fait qu'un chemin d'augmentation $C$ alterne la traversée d'arêtes dans $M$ et dans $E\minus M$, et comme il commence et se termine par des arêtes dans $E\minus M$ (puisque ses extrémités sont libres pour $M$), il est alors nécessaire que la longueur du chemin $C$ soit impaire. Par bipartisme de $G$, on en déduit que les extrémités de $C$ sont nécessairement un élément de $X$ et un élément de $Y$.
 
 Si $C$ est un chemin d'augmentation pour $M$ dans $G$, en identifiant $C$ avec l'ensemble des arêtes qu'il visite, on peut définir l'ensemble $M' := M \oplus C$ (où $\oplus$ désigne la différence symétrique entre ensembles), et remarquer qu'alors $M'$ est encore un couplage de $G$ : tout sommet de $(V, M')$ est en effet encore de degré inférieur ou égal à 1, et bien sûr $M' \subseteq E$.
 
 Par ailleurs, $M'$ contient exactement une arête de plus que le couplage $M$. À partir du couplage trivial $M_0 = \empty$, on peut ainsi construire une suite finie $(M_0,\dots,M_s)$ de couplages de $G$ qui soit strictement croissante pour le cardinal : l'élément $M_{i+1}$ est définit comme la différence symétrique de $M_i$ avec un chemin d'augmentation pour $M_i$ dans $G$. La suite s'arête lorsqu'il ne reste plus de chemins d'augmentation à trouver pour le dernier élément de la suite.
 
 Afin de prouver que le dernier couplage $M_s$ de la suite, obtenu selon cette méthode des chemins d'augmentation, est un couplage maximum de $G$, (c'est-à-dire que pour tout autre couplage $M$ de $G$, on aurait $\abs{M_s} \geq \abs M$), nous avons le théorème suivant.
 
 \SEP\jtheoz{Berge 1957}
    Un couplage $M$ est maximum si et seulement si $G$ ne contient pas de chemins d'augmentation pour $M$.
 \SEP\jpreuve
    Comme montré précédemment, s'il existe un chemin d'augmentation $C$ pour $M$, alors $C \oplus M$ est un couplage contenant une arête de plus que $M$, ce qui implique que $M$ ne peut être maximum.
    
    Réciproquement, supposons $M$ non maximum. Soit $M\sun$ un couplage maximum dans $G$. On doit donc avoir $\lam := \abs {M\sun} - \abs M > 0$. Définissons $S := M \oplus M\sun$. $S$ est l'ensemble des arêtes qui ne sont pas communes à $M$ et $M\sun$ : c'est l'union disjointe de $S \cap M\sun = M\sun \minus M$ et $S \cap M = M \minus M\sun$, et on doit avoir $\abs{M\sun \minus M} = \abs{M \minus M\sun}+\lam$.
    
    Soit $x \in V$. Dans le graphe partiel $(V, S)$, si $x$ est l'extrémité d'une arête de $M$, alors il ne peut être l'extrémité d'une autre arête de $M$, puisque $M$ est un couplage. De la même façon, $x$ est l'extrémité d'au plus une arête de $M\sun$. Par suite, tous les sommets de $(V,S)$ sont de degré inférieurs ou égaux à 2.
    
    On en déduit que $S$ contient un ensemble de cycles et de chemins élémentaires maximaux de $G$ (que l'on ne peut pas prolonger dans $S$) deux à deux sommet-disjoints, et où il y a toujours alternance entre une arête de $M$ et une arête de $M\sun$ dans chacun de ces cycles et chemins.
    
    Les cycles sont donc tous de longueur paire, et contiendront aussi autant d'arêtes de $M$ que d'arêtes de $M\sun$. De même, tout chemin maximal de $S$ de longueur paire contiendra autant d'arêtes de $M$ que d'arêtes de $M\sun$.
    
    Comme il est supposé y avoir $\lam$ arêtes de plus dans $S \cap M\sun$ que dans $S \cap M$, cela implique forcément qu'il existe dans $S$ au moins un chemin $C$ ayant plus d'arêtes provenant de $M\sun$ que d'arêtes provenant de $M$. Un tel chemin est alors de longueur impaire, començant et se terminant par deux arêtes de $M\sun \minus M$, donc par deux sommets libres par rapport à $M$, car si les extrémités de $C$ n'étaient pas libre pour $M$, le chemin $C$ pourrait être prolongé d'au moins une arête, et ne serait donc pas maximal (ou bien ce serait un cycle, ce qui a déjà été exclus). Par suite, $C$ est un chemin d'augmentation pour $M$ dans $G$, ce qui achève la preuve de la réciproque.
    
    Précisons un point qui sera utile pour prouver le théorème qui va suivre : il n'existe aucun chemin maximal dans $S$ contenant moins d'arêtes de $M\sun$ que d'arêtes de $M$ : si un tel chemin existait, ce serait un chemin d'augmentation pour $M\sun$, ce qui est impossible du fait que $M\sun$ est un couplage maximum pour $G$.
 \qed\SEP
 \jtheoz{Hopcroft-Karp} On reprend les hypothèses et notations du théorème précédent. Alors, il existe, dans $S = M\sun \oplus M$, exactement $\lam = \abs {M\sun} - \abs M > 0$ chemins d'augmentation pour $M$ qui soient deux-à-deux sommet-disjoints.
 \SEP\jpreuve
    Il suffit de voir que chaque chemin d'augmentation pour $M$ dans $S$ contient exactement une arête de plus provenant de $M\sun$, que d'arêtes provenant de $M$. Il existe donc nécessairement $\lam$ chemins d'augmentation disjoints dans $S$ pour justifier que $M\sun$ contienne $\lam$ arêtes de plus que $M$, tout cela en tenant compte que, comme montré précédemment, les chemins d'augmentation pour $M$ sont les seules parties de $S$ qui influent sur la différence de taille entre $M$ et $M\sun$.
 \qed \SEP
 
 Par application du théorème précédent, $M_s$ étant le dernier couplage de la suite, comme aucun chemin d'augmentation n'existe pour celui-ci, $M_s$ est donc nécessairement un couplage maximum. Cela achève la preuve de la validité de la méthode de résolution du problème de couplage maximum par chemins d'augmentation.
 
 \subsection{L'algorithme}
 
 On améliore l'algorithme esquissé précédemment de la manière suivante : au lieu d'utiliser un seul chemin d'augmentation quelconque à chaque étape, on récupère un ensemble maximal $K$ aussi grand que possible de chemins d'augmentation de $M$ qui soient tous de la longueur $k$ la plus courte possible, et deux à deux sommet-disjoints. On utilise alors chaque élément de $K$ pour augmenter $M$.
 
 A noter que $K$ n'a besoin d'être \emph{maximal} que au sens de l'inclusion, et non pas \emph{maximum} au sens du cardinal, ce qui serait autrement plus dur à construire. Concrètement, s'il n'existe pas d'autre chemin de longueur $k$ et deux à deux sommet-disjoint de tous les éléments de $K$, alors on considèrera que $K$ est maximal.
 
 
 \SEP \emph{\bfseries Algorithme Hopcroft-Karp.}
    
    Entrée : Graphe biparti simple non orienté $G = (V,E) = (X \cup Y, E)$.
    
    Sortie : Couplage maximum $M$ de $G$.
    \bigskip
    
    $M \leftarrow \empty$

    
    \code{Faire \{}
    
    \hspace{2em} $K \leftarrow \set{C_1,\dots,C_n}$
    
        
    \hspace{2em} \code{// $\forall i,j$, $C_i$ est un chemin d'augmentation pour $M$ de longueur minimale $k$}
    
    \hspace{2em} \code{// et $C_i \cap C_j = \empty$ par sommets et arêtes}
    
    \hspace{2em}  \code{// et $K$ est maximal au sens de l'inclusion.}
    
    \hspace{2em} $ M \leftarrow M \oplus \bigcup_{C \in K} C = (((M \oplus C_1) \oplus C_2) \cdots ) \oplus C_n$
    
    \code{\} Tant que $K \neq \empty$.} \code{// tant qu'il reste des chemins d'augmentation à trouver}
    
    \code{Retourner $M$.}
 \SEP
 
  \subsection{Mise en Œuvre Concrète}
 
 On divise le travail de la construction progressive du couplage $M$, en \emph{étapes}, toutes identiques, et qui ne s'arrêtent que lorsqu'il n'y a plus rien à faire. Chaque étape construit $K$ et utilise ses éléments pour augmenter $M$. Une étape correspond en fait à un passage dans la boucle \code{tant que ... faire} du pseudo-code ci-dessus. À chaque étape, il y a deux phases. Définissons $L \subseteq X$ l'ensemble des sommets de $X$ libres pour le couplage $M$.
 
 La phase \#1 consiste à récupérer la longueur $k$ du plus petit chemin d'augmentation que l'on puisse trouver pour $M$ dans $G$. Cela se fait par un parcours en largeur qui commence par traiter tous les sommets de $L$, puis qui se poursuit en suivant la règle que les arêtes explorées doivent être dans $M$ et dans $E \minus M$ en alternance. En fait, comme toutes les arêtes du niveau $0$ sont dans $E\minus M$ (car toutes les racines $L$ du parcours sont libres pour $M$), chaque niveau de profondeur du parcours est intégralement inclus dans $M$, puis dans $E \minus M$, en alternance.
 
 On explore donc le graphe $G$ en s'arrêtant à la profondeur $k$, c'est-à-dire à la profondeur où on trouve le premier sommet libre de $Y$, donc où l'on trouve le plus petit chemin d'augmentation pour $M$ dans $G$. Cependant on ne s'arrête pas avant d'avoir traité tous les sommets à cette profondeur. Lors de ce parcours largeur, on mémorise aussi, pour chaque sommet $s$ visité, la profondeur $p(s)$ à laquelle on l'a visité. On interdit bien sûr les revisites, car les chemins d'augmentations sont élémentaires.

 La phase \#2 consiste en la recherche des éléments de $K$, et leur utilisation pour augmenter $M$. Chaque élément de $K$ peut être trouvé par un parcours en profondeur de $G$ à partir d'un sommet de $L$ particulier, qui sera l'une de ses extrémités. Durant chacun de ces parcours, on alterne bien sûr la traversée d'arêtes dans $M$ et dans $E\minus M$. On s'arrange également pour que tout sommet $v$ visité \emph{à partir} d'un sommet $s$ soit de profondeur $p(v) = p(s) + 1$ (telle que mesurée durant la phase \#1), ce qui nous assure de trouver le chemin le plus court possible (de longueur $k$). On interdit de plus la revisite d'un sommet de $G$ durant toute la phase \#2 : cela nous assure que les chemins de $K$ trouvés sont bien deux à deux sommet-disjoints.
 
 C'est un choix justifié du fait que si un sommet $v$ est visité par un parcours profondeur à partir d'un certain sommet libre de $X$, mais si $v$ n'est pas retenu pour faire partie d'un chemin de $K$, alors les autres parcours le rejetteront tout autant, car il ne permet pas, à sa profondeur $p(v)$, de rejoindre un élément libre de $Y$ de manière compatible avec un chemin d'augmentation de $M$ de longueur $k$.
 
 L'augmentation de $M$ en utilisant chacun des chemins de $K$ se fait à la volée durant la remontée de chaque parcours en profondeur qui a trouvé le chemin en question.

 \subsection{Complexité}
 
  La complexité maximale de chaque phase \#1 et \#2, est donc la somme de celle d'un parcours largeur et d'un parcours profondeur, tous deux généralisés -- l'absence de revisites durant chaque phase justifiant cette évaluation. Chaque étape de l'algorithme a donc une complexité temporelle qui se mesure donc en $O(2(n+m)) = O(n + m)$.
 
 Comme le graphe est biparti toutes les arêtes partent de $X$ et atterrissent dans $Y$ donc on a finalement
 \[ m \leq \abs {X \times  Y} = \abs X \times \abs Y \leq \paren{\dfrac n 2}^2 = \dfrac {n^2} 4 \]
 du fait que $n = \abs X + \abs Y$ et que l'on a, pour tous $x,y \in \R$,
 \[0 \leq (x - y)^2  = (x + y)^2 - 4xy \donc xy \leq \frac{(x + y)^2}{4} \]
 Donc $m = O(n^2)$, par suite $m$ l'emporte sur $n$ dans le pire cas, et la complexité d'une étape de l'algorithme est donc $O(m)$.
 
 Il peut être montré qu'à chaque étape, la longueur minimale $k$ pour laquelle on puisse trouver un chemin d'augmentation, augmente d'au moins 1. Ce résultat est non trivial, voire même difficile. On peut en trouver une preuve dans [Meena Mahajan, ``Matchings in Graphs'', 2010].
 
 Après l'étape $\sqrt n$, tous les chemins d'augmentation que l'on peut trouver seront donc de longueur au moins égale à $\sqrt n$. Soit $M$ le couplage obtenu à ce moment-là de l'algorithme, $M\sun$ un couplage maximum pour $G$, et $S := M \oplus M\sun$.
 
 $S$ contient entre autres un certain nombre de chemins d'augmentation de $M$, donc nécessairement tous de longueur supérieure ou égale à $\sqrt n$. Comme ils sont deux-à-deux sommet-disjoints (\cf le théorème de Berges), chaque chemin d'augmentation passe par un peu plus de $\sqrt n$ sommets qu'il ne partage pas avec les autres chemins dans $S$, et alors il ne peut donc y avoir plus de $\sqrt n$ chemins d'augmentation distincts dans $S$, puisque le nombre total de sommets disponibles est limité à $n$.
 
 D'après le théorème de Hopcroft-Karp, le nombre de chemins d'augmentation dans $S$ est égal à la différence $\abs {M\sun} - \abs M$. D'où ici, $\abs {M\sun} - \abs M \leq \sqrt n$. Comme chaque étape de l'algorithme trouve au moins un chemin d'augmentation, à chaque étape $M$ croît en cardinal par au moins 1. Par suite, à l'étape $2 \sqrt n$ (ou avant si l'algorithme termine plus tôt), on a forcément $\abs M = \abs {M\sun}$, et $M$ est donc un couplage maximum, comme recherché.
 
 Par suite, l'algorithme de Hopcroft-Karp se termine donc toujours après $O(2\sqrt n) = O(\sqrt n)$ étapes, chaque étape ayant une complexité de $O(m)$, d'où la complexité finale de Hopcroft-Karp qui est bien $O(m \sqrt n) = O(n^2 \sqrt n) = O(n^{2.5})$.
 
\end{document}
